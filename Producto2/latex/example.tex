\section{Mostrar inicialmente un menú}
	Inicialmente mostramos el menú con las tres funciones implementadas en la función \textit{show\_menu}, esta función se encuentra en la aplicación de consola en el proyecto principal de la solución, hace uso de librerías estáticas para el uso de funciones de entrada/salida en consola o funciones de archivo.
	\begin{sourcecodep}[]{c}{basicstyle={\fontsize{12}{12}\selectfont\ttfamily},firstnumber=1}{}
/*
	Función para construir el menú de la aplicación de consola.
*/
void show_menu() {
	print_msg("============================================");
	print_msg("Especifica que quieres hacer por su numero:");
	print_msg("1. Mostrar el contenido de un archivo.");
	print_msg("2. Guardar como un archivo.");
	print_msg("3. Chequear el contenido de un archivo.");
	// Preguntamos por la opción a ejecutar.
	int selected = ask_for_integer();
	print_msg_int("Has seleccionado: ", selected);
	switch (selected)
	{
		case 1:
			// Mostramos el contenido de un archivo
			show_content_of_file();
			break;
		case 2:
			// Guardamos como, un archivo
			save_file_as();
			break;
		case 3:
			// Chequeamos el contenido de un archivo
			check_content_of_file();
			break;
		default:
			print_msg("La opci\242n seleccionada no es correcta.");
			// Volvemos a mostrar el menú si la opción seleccionada no es correcta
			show_menu();
	}
}\end{sourcecodep}
\newpage
\section{Diseñar una función que muestre el contenido de un archivo}
	\begin{sourcecodep}[]{c}{basicstyle={\fontsize{12}{12}\selectfont\ttfamily},firstnumber=1}{}
/*
	Función para mostrar el contenido de un archivo
*/
void show_content_of_file() {
	print_msg("============================================");
	print_msg("Introduce la ruta del archivo a leer:");
	const char * path[255];
	// Obetenemos mediante consola, la ruta del archivo
	read_string(path, 255);
	
	// Abrimos el archivo
	FILE * ptr_file = open_file(path);
	// Si el archivo no existe retornamos
	if (ptr_file == NULL)
	{
		print_msg("El archivo no existe.");
		return;
	}
	print_msg("========== Contenido del archivo: ==========");
	char c;
	c = fgetc(ptr_file);
	// Mientras que no lleguemos al final del archivo
	while (c != EOF)
	{
		// Escribimos en consola cada carácter del archivo
		print_char(c);
		c = fgetc(ptr_file);
	}
	print_line_break();
	// Cerramos el archivo
	fclose(ptr_file);
}\end{sourcecodep}
\newpage
\section{Diseñar una función que implemente el guardado como de un archivo}
	\begin{sourcecodep}[]{c}{basicstyle={\fontsize{10}{10}\selectfont\ttfamily},firstnumber=1}{}
/*
	Función para guardar como, un archivo
*/
void save_file_as() {
	print_msg("============================================");
	print_msg("Introduce la ruta del archivo a copiar:");
	const char * path[255];
	// Obetenemos mediante consola, la ruta del archivo origen
	read_string(path, 255);
	
	// Abrimos el archivo
	FILE * ptr_file = open_file(path);
	// Si el archivo no existe retornamos
	if (ptr_file == NULL)
	{
		print_msg("El archivo no existe.");
		return;
	}
	
	print_msg("Introduce la ruta del archivo destino:");
	const char * path_cp[255];
	// Obetenemos mediante consola, la ruta del archivo destino
	read_string(path_cp, 255);
	
	// Creamos el archivo
	FILE * ptr_file_cp = create_file(path_cp);
	// Si ha habido error al crear el archivo retornamos
	if (ptr_file_cp == NULL)
	{
		print_msg("Error al crear el archivo.");
		return;
	}
	
	print_msg("Archivo destino creado.");
	
	char c;
	c = fgetc(ptr_file);
	// Mientras que no lleguemos al final del archivo
	while (c != EOF)
	{
		// Copiamos cada carácter del archivo origen, en el archivo destino
		fputc(c, ptr_file_cp);
		c = fgetc(ptr_file);
	}
	print_line_break();
	// Cerramos el archivo origen
	fclose(ptr_file);
	// Borramos el archivo origen
	if (remove(path) != 0)
		print_msg("No se pudo eliminar el archivo de origen.");
	else
		print_msg("Archivo origen eliminado.");
	// Cerramos el archivo destino
	fclose(ptr_file_cp);
}\end{sourcecodep}
\newpage
\section{Diseñar una función que chequee el contenido de un archivo}
	\begin{sourcecodep}[]{c}{basicstyle={\fontsize{10}{10}\selectfont\ttfamily},firstnumber=1}{}
/*
Función para chequear el contenido de un archivo
*/
void check_content_of_file() {
	print_msg("============================================");
	print_msg("Introduce la ruta del archivo a buscar:");
	const char * path[255];
	// Obetenemos mediante consola, la ruta del archivo
	read_string(path, 255);
	
	// Abrimos el archivo
	FILE * ptr_file = open_file(path);
	// Si el archivo no existe retornamos
	if (ptr_file == NULL)
	{
		print_msg("El archivo no existe.");
		return;
	}
	
	print_msg("Introduce la palabra o frase a buscar:");
	const char * word[255];
	// Obetenemos mediante consola, la palabra o frase a buscar
	read_string(word, 255);
	
	char c;
	c = fgetc(ptr_file);
	bool found = FALSE;
	char line[1024];
	
	// Por cada linea del archivo
	while (fgets(line, 1024, ptr_file) != NULL) {
		// Comprobamos si contiene la palabra o frase a buscar
		if ((strstr(line, word)) != NULL) {
			// Si la contiene, seteamos como encontrada y rompemos el bucle
			found = TRUE;
			break;
		}
	}
	
	// Cerramos el archivo
	fclose(ptr_file);
	
	// Mostramos en consola el resultado
	if (found) {
		print_msg("Palabra encontrada!");
	}
	else {
		print_msg("Palabra no encontrada.");
	}
}\end{sourcecodep}
\newpage
\section{Realizar la aplicación modularizada}
	Para llevar a cabo la modularización de la aplicación, se ha distribuido el código en una aplicación de consola principal y dos librerías estáticas, una para las funciones de consola (librería \textit{consolelib}), para el uso de funciones de entrada/salida en consola y otra para las funciones de archivos (librería \textit{filelib}).
	
	\insertimage[]{img/solucion}{scale=1}{Solución en Visual Studio.}