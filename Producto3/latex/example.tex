\section{Función main, entry point de nuestra aplicación.}
	\begin{sourcecodep}[]{c}{basicstyle={\fontsize{12}{12}\selectfont\ttfamily},firstnumber=1}{}
		/*
		Función main, entry point de la aplicación de consola.
		*/
		int main()
		{
			printf("%s\n", "============================================");
			printf("%s\n", "Introduce la ruta del archivo hosts.");
			const char * path[255];
			// Obtenemos mediante consola, la ruta del archivo
			read_string(path, 255);
			// Mostramos el contenido del archivo por consola
			show_content_of_file(path);
			
			printf("%s\n", "============================================");
			printf("%s\n", "Introduce la ruta del archivo pares.");
			const char * diff_path[255];
			// Obtenemos mediante consola, la ruta del archivo
			read_string(diff_path, 255);
			// Hacemos la comparación de los dos archivos y copiamos la diferencia
			diff_two_files(path, diff_path);
			// Mostramos el contenido del archivo por consola
			show_content_of_file(path);
			
			system("pause");
			return 0;
		}\end{sourcecodep}
	Desde la función main, como podemos ver, llamamos a las diferentes funciones para realizar las distintas tareas que requerimos realizar sobre los distintos ficheros.
	
	\begin{itemize}
		\item Pedimos la ruta del archivo hosts
		\item Mostramos el contenido del archivo hosts
		\item Pedimos la ruta del archivo con los pares
		\item Hacemos la comparación de los dos archivos y aplicamos las diferencias al archivo hosts
		\item Mostramos el contenido del archivo hosts
	\end{itemize}
\newpage
\section{Función para mostrar el contenido de un archivo.}
	\begin{sourcecodep}[]{c}{basicstyle={\fontsize{12}{12}\selectfont\ttfamily},firstnumber=1}{}
		/*
		Función para mostrar el contenido de un archivo
		*/
		void show_content_of_file(const char * path) {
			// Abrimos el archivo
			FILE * ptr_file = open_file(path);
			// Si el archivo no existe retornamos
			if (ptr_file == NULL)
			{
				printf("%s\n", "El archivo no existe.");
				return;
			}
			printf("%s%s%s\n", "========== Contenido del archivo: ", path, " ==========");
			char c;
			c = fgetc(ptr_file);
			// Mientras que no lleguemos al final del archivo
			while (c != EOF)
			{
				// Escribimos en consola cada carácter del archivo
				printf("%c", c);
				c = fgetc(ptr_file);
			}
			printf("\n");
			// Cerramos el archivo
			fclose(ptr_file);
		}\end{sourcecodep}
	Esta función está diseñada para mostrar todo el contenido de un archivo por consola, carácter por carácter. Nos sirve para mostrar los cambios realizados en el archivo hosts y verificar que estos están correctos.
\newpage
\section{Función para chequear el contenido de un archivo.}
	\begin{sourcecodep}[]{c}{basicstyle={\fontsize{12}{12}\selectfont\ttfamily},firstnumber=1}{}
		/*
		Función para chequear el contenido de un archivo
		*/
		bool check_content_of_file(const char * path, const char * find) {
			
			// Abrimos el archivo
			FILE * ptr_file = open_file(path);
			// Si el archivo no existe retornamos
			if (ptr_file == NULL)
			{
				printf("%s\n", "El archivo no existe.");
				return false;
			}
			
			bool found = false;
			char * line[1024];
			
			// Por cada linea del archivo
			while (fgets(line, 1024, ptr_file) != NULL) {
				// Comprobamos si contiene la palabra o frase a buscar
				if ((strstr(line, find)) != NULL) {
					// Si la contiene, seteamos como encontrada y rompemos el bucle
					found = true;
					break;
				}
			}
			
			// Cerramos el archivo
			fclose(ptr_file);
			
			// Retornamos found
			return found;
		}\end{sourcecodep}
	Con esta función podemos verificar que un archivo contiene cierta palabra o conjunto de palabras. Nos es útil para saber si el archivo hosts contiene alguno de los pares que debemos incorporar desde el archivo de pares.
\newpage
\section{Función para comparar dos archivos, agregar las diferencias a un archivo temporal y aplicarlos al archivo principal.}
	\begin{sourcecodep}[]{c}{basicstyle={\fontsize{9}{9}\selectfont\ttfamily},firstnumber=1}{}
		/*
		Función para comparar dos archivos, agregar las diferencias a un archivo temporal y aplicarlos al archivo principal
		*/
		void diff_two_files(const char * path, const char * diff_path) {
			// Abrimos el archivo diff
			FILE * ptr_file_diff = open_file(diff_path);
			// Si el archivo no existe retornamos
			if (ptr_file_diff == NULL)
			{
				printf("%s\n", "El archivo no existe.");
				return;
			}
			
			// Creamos el archivo temporal
			FILE * ptr_file_tmp = tmpfile();
			// Si el archivo no existe retornamos
			if (ptr_file_tmp == NULL)
			{
				printf("%s\n", "Error al crear el archivo temporal.");
				return;
			}
			
			char * line[1024];
			
			// Por cada linea del archivo diff
			while (fgets(line, 1024, ptr_file_diff) != NULL) {
				// Comprobamos si contiene el contenido de la linea
				if (!check_content_of_file(path, line)) {
					// La añadimos al archivo temporal
					fputs(line, ptr_file_tmp);
				}
			}
			// Posicionamos el puntero al principio
			fseek(ptr_file_tmp, 0, SEEK_SET);
			
			// Abrimos el archivo
			FILE * ptr_file = open_file(path);
			// Si el archivo no existe retornamos
			if (ptr_file == NULL)
			{
				printf("%s\n", "El archivo no existe.");
				return;
			}
			// Posicionamos el puntero al final
			fseek(ptr_file, 0, SEEK_END);
			fputs("\n", ptr_file);
			// Por cada linea del archivo 
			while (fgets(line, 1024, ptr_file_tmp) != NULL) {
				// La añadimos al archivo
				fputs(line, ptr_file);
			}
			
			// Cerramos los archivos
			fclose(ptr_file);
			fclose(ptr_file_diff);
			fclose(ptr_file_tmp); //El archivo temporal se borra solo al cerrarlo
		}\end{sourcecodep}
	Con esta función realizamos la tarea más importante de la aplicación, incorporar al archivo hosts los pares del archivo de pares. Para esto seguimos una serie de acciones:
	\begin{itemize}
		\item Abrimos el archivo de pares
		\item Creamos el archivo temporal donde guardaremos los pares que no contenga el archivo hosts
		\item Comprobamos que el archivo hosts contenga los pares del archivo de pares, los que no contiene los añadimos al archivo temporal
		\item Abrimos el archivo hosts
		\item Añadimos al archivo hosts cada par guardado en el archivo temporal
		\item Cerramos los archivos
		\item El archivo temporal es automáticamente borrado por el propio sistema ya que usamos la función \href{https://www.tutorialspoint.com/c_standard_library/c_function_tmpfile.htm}{\textit{tmpfile}}
	\end{itemize}

\section{Salida}
	\insertimage[]{img/salida}{scale=0.45}{}